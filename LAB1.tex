ddddddd
ФИТУШНИИИr
\usepackage{multicol}
\usepackage{amsmath}
\usepackage{amssymb}
\usepackage[left=20mm, top=17mm, right=16mm, bottom=18mm, ]{geometry}
\usepackage{enumitem}
\usepackage{fancyhdr}
\usepackage[russian]{babel}
\setlength{\columnsep}{0.5cm}
\graphicspath{{files/}}
\pagestyle{fancy}
\fancyhf{}
\renewcommand{\headrulewidth}{0pt}
\fancyfoot[C]{\textbf{\thepage}}

\setcounter{page}{27}
\usepackage{setspace}
% выравнивания текста по ширине в документе.
\linespread{0.3}
\begin{document}

\setcounter{figure}{2}

\begin{figure}
    \centering
   
\includegraphics[width=18cm]{1.png}

    \caption{ Example of method interpretation by the ostis-systems collective}
    
  
    \end{figure}
\setlength{\parskip}{0pt}% между абзацами
\begin{multicols}{2}
\setlist{noitemsep} % чтоб не было пустых строк

 \footnotesize%размер шрифта
 
\noindent
\leftskip=8mm 
\textit {Proceedings of the Conference. In 2 volumes, Moscow, December
21-23, 2022. Volume 2.]} National Research University "MPEI", 2022, pp. 275–291.

\begin{itemize}
  \renewcommand{\labelitemi}{[2]}
  \item L. Cao, “Decentralized ai: Edge intelligence and smart
blockchain, metaverse, web3, and desci,
 \textit{,” IEEE Intelligent Systems, }. vol. 37, no. 3, p. 6–19, May 2022. [Online]. Available:
http://dx.doi.org/10.1109/MIS.2022.3181504
\renewcommand{\labelitemi}{[3]}
		\item D. Ye, M. Zhang, and A. V. Vasilakos, “A survey of
self-organization mechanisms in multiagent systems,  \textit{” IEEE
Transactions on Systems, Man, and Cybernetics: Systems}. vol. 47, no. 3, p. 441–461, Mar. 2017. [Online]. Available:
http://dx.doi.org/10.1109/TSMC.2015.2504350
\renewcommand{\labelitemi}{[4]}	
\item   V. Golenkov, Ed.,\textit{ Tehnologija kompleksnoj podderzhki
zhiznennogo cikla semanticheski sovmestimyh intellektual’nyh
komp’juternyh sistem novogo pokolenija [Technology of complex
life cycle support of semantically compatible intelligent computer
systems of new generation ].} Bestprint, 2023.
\renewcommand{\labelitemi}{[5]}
\item  A. Kolesnikov, \textit{Gibridnye intellektual’nye sistemy: Teoriya i
tekhnologiya razrabotki [Hybrid intelligent systems: theory and
technology of development]} A. M. Yashin, Ed. SPb.: Izd-vo
SPbGTU, 2001.
\renewcommand{\labelitemi}{[6]}
\item S. Y. Mikhnevich and A. A. Tsezhar, “Evolution of the
concept of interoperability of open information systems,”
\textit{Digital Transformation,} vol. 29, no. 2, p. 60–66, Jun. 2023.
[Online]. Available: http://dx.doi.org/10.35596/1729-7648-2023-
29-2-60-66
\renewcommand{\labelitemi}{[7]}
\item  D. Shunkevich, “Principles of problem solving in distributed
teams of intelligent computer systems of a new generation, \textit{” Open
semantic technologies for intelligent systems, } no. 7, pp. 115–120,
2023.
\renewcommand{\labelitemi}{[8]}
\item A. Zagorskiy, “Principles for implementing the ecosystem of
next-generation intelligent computer systems,\textit{” Open semantic
technologies for intelligent systems,}no. 6, pp. 347–356, 2022.
\renewcommand{\labelitemi}{[9]}
\item  V. Golenkov, N. Guliakina, and D. Shunkevich, \textit{Otkrytaja
tehnologija ontologicheskogo proektirovanija, proizvodstva
i jekspluatacii semanticheski sovmestimyh gibridnyh
intellektual’nyh komp’juternyh sistem [Open technology of
ontological design, production and operation of semantically
compatible hybrid intelligent computer systems],} V. Golenkov,
Ed. Minsk: Bestprint [Bestprint], 2021
\renewcommand{\labelitemi}{[10]}
\item  V. Tarasov,\textit{ Ot mnogoagentnykh sistem k intellektual’nym
organizatsiyam [From multi-agent systems to intelligent
organizations]}. M.: Editorial URSS, 2002, (in Russian).
\renewcommand{\labelitemi}{[11]}
\item D. Shunkevich, “Hybrid problem solvers of intelligent computer
systems of a new generation,\textit{” Open semantic technologies for
intelligent systems,} no. 6, pp. 119–144, 2022.
\renewcommand{\labelitemi}{[12]}
\item G. Di Marzo Serugendo, M.-P. Gleizes, and A. Karageorgos,
“Self-organization in multi-agent systems,”\textit{ The Knowledge Engineering Review,} vol. 20, no. 2, p. 165–189, Jun. 2005. [Online].
Available: http://dx.doi.org/10.1017/S0269888905000494
\renewcommand{\labelitemi}{[13]}
\item   ——, “Selforganisation and emergence in multiagent systems: An
overview,” \textit{Informatica}, vol. 30, no. 1, p. 45–54, 2006.
\renewcommand{\labelitemi}{[14]}
\item Q.-N. N. Tran and G. Low, “MOBMAS: A methodology for
ontology-based multi-agent systems development,\textit{” Information
and Software Technology}, vol. 50, no. 7–8, p. 697–722, Jun.
2008. [Online]. Available: http://dx.doi.org/10.1016/j.infsof.2007.
07.005
\renewcommand{\labelitemi}{[15]}
\item
A. Fayans and V. Kneller, “About the ontology of task
types and methods of their solution,\textit{” Ontology of designing},
vol. 10, no. 3, pp. 273–295, Oct. 2020. [Online]. Available:
https://doi.org/10.18287/2223-9537-2020-10-3-273-295
\renewcommand{\labelitemi}{[16]}
\item
V. Golenkov, N. Gulyakina, I. Davydenko, and D. Shunkevich,
“Semanticheskie tekhnologii proektirovaniya intellektual’nyh
sistem i semanticheskie associativnye komp’yutery [Semantic
technologies of intelligent systems design and semantic
associative computers],”\textit{ Otkrytye semanticheskie tehnologii
proektirovanija intellektual’nyh sistem [Open semantic
technologies for intelligent systems]}, pp. 42–50, 2019.
\renewcommand{\labelitemi}{[17]}
\item
V. Gorodetskii, “Samoorganizacija i mnogoagentnye sistemy. i.
modeli mnogoagentnoj samoorganizacii [self-organization and
multi-agent systems. i. models of multi-agent self-organization],”
\textit{Izvestiya RAN. Teoriya i sistemy upravleniya [Proceedings of the RAS. Theory and control systems]}, no. 2, pp. 92–120, 2012, (in
Russian).
\renewcommand{\labelitemi}{[18]}
\item P. S. Sapaty,\textit{ Spatial Grasp as a Model for Space-based Control}
\newpage
    \textit{and Management Systems.} CRC Press, Apr. 2022. [Online].
Available: http://dx.doi.org/10.1201/9781003230090

\renewcommand{\labelitemi}{[19]}
\item ——,\textit { The Spatial Grasp Model: Applications and
 Investigations of Distributed Dynamic Worlds} Emerald
Publishing Limited, Jan. 2023. [Online]. Available:
http://dx.doi.org/10.1108/9781804555743
\renewcommand{\labelitemi}{[20]}
\item A. A. Letichevsky, O. A. Letychevskyi, and V. S. Peschanenko,
\textit{Insertion Modeling System.} Springer Berlin Heidelberg, 2012, p.
262–273.
\renewcommand{\labelitemi}{[21]}
\item A. Letichevsky, Yu. Kapitonova, V. Volkov, V. Vyshemirsky, and
A. Letichevsky (j.), “Insercionnoe programmirovanie [Insertion
programming],”\textit { Kibernetika i sistemnyj analiz [Cybernetics and
system analysis],} no. 1, pp. 19–32, 2003.
\renewcommand{\labelitemi}{[22]}
\item  I. Davydenko, “Semantic models, method and tools of knowledge bases coordinated development based on reusable components,” in \textit { Otkrytye semanticheskie tehnologii proektirovanija
intellektual’nyh sistem [Open semantic technologies for intelligent
systems]}, V. Golenkov, Ed., BSUIR. Minsk , BSUIR, 2018, pp.
99–118.
\renewcommand{\labelitemi}{[23]}
\item M. Kovalev, “Convergence and integration of artificial neural
networks with knowledge bases in next-generation intelligent
computer systems,”\textit { Open semantic technologies for intelligent
systems}, no. 6, pp. 173–186, 2022.
\renewcommand{\labelitemi}{[24]}
\item
M. Orlov, “Control tools for reusable components of intelligent
computer systems of a new generation,”\textit { Open semantic technologies for intelligent systems}, no. 7, pp. 191–206, 2023.
\renewcommand{\labelitemi}{[25]}
\item D. Pospelov,\textit {  Situacionnoe upravlenie. Teorija i praktika
[Situational management. Theory and practice]}. M.: Nauka,
1986
\renewcommand{\labelitemi}{[26]}
\item
] L. Cao, “In-depth behavior understanding and use: The
behavior informatics approach,”\textit {  Information Sciences}, vol.
180, no. 17, pp. 3067–3085, Sep. 2010. [Online]. Available:
https://doi.org/10.1016/j.ins.2010.03.025
\renewcommand{\labelitemi}{[27]}
\item
{[27]} L. Cao, T. Joachims, C. Wang, E. Gaussier, J. Li, Y. Ou,
D. Luo, R. Zafarani, H. Liu, G. Xu, Z. Wu, G. Pasi, Y. Zhang,
X. Yang, H. Zha, E. Serra, and V. Subrahmanian, “Behavior
informatics: A new perspective,”\textit { IEEE Intelligent Systems,}
vol. 29, no. 4, pp. 62–80, Jul. 2014. [Online]. Available:
https://doi.org/10.1109/mis.2014.60
\renewcommand{\labelitemi}{[28]}
\item
M. Pavel, H. B. Jimison, I. Korhonen, C. M. Gordon, and
N. Saranummi, “Behavioral informatics and computational
modeling in support of proactive health management and
care,”\textit {IEEE Transactions on Biomedical Engineering,} vol. 62,
no. 12, pp. 2763–2775, Dec. 2015. [Online]. Available:
https://doi.org/10.1109/tbme.2015.2484286
\renewcommand{\labelitemi}{[29]}
\item
G. S. Al’tshuller,\textit {Najti ideju: Vvedenie v TRIZ — teoriju reshenija
izobretatel’skih zadach, 3-e izd. [Find an idea: An introduction
to TRIZ - the theory of inventive problem solving, 3rd ed.].} M.:
Al’pina Pablisher, 2010.
\renewcommand{\labelitemi}{[30]}
\item G. P. Shhedrovickij,\textit { Shema mysledejatel’nosti – sistemnostrukturnoe stroenie, smysl i soderzhanie [Scheme of mental
activity – system-structural structure, meaning and content].} M.:
Shk. kul’t. pol., 1995.
\end{itemize}
\columnbreak
\begin{center}
\begin{onehalfspace}
 \textbf{\Large{ПРИНЦИПЫ
ДЕЦЕНТРАЛИЗОВАННОГО РЕШЕНИЯ
ЗАДАЧ В РАМКАХ ЭКОСИСТЕМЫ
ИНТЕЛЛЕКТУАЛЬНЫХ
КОМПЬЮТЕРНЫХ СИСТЕМ НОВОГО
    ПОКОЛЕНИЯ }}
\end{onehalfspace}
 \end{center}
 \setlength{\parskip}{10pt}
\begin{center}
\textup{\Large{Шункевич Д.В.}}\\
 \end{center}
 \large {В работе рассмотрены принципы децентрализованного решения задач в рамках экосистемы интеллектуальных компьютерных систем нового поколения, в
частности рассмотрена архитектура такой экосистемы
с точки зрения организации процесса решения задач,
выделены роли систем, участвующих в процессе решения задач. Уточнены принципы формирования коллектива систем, участвующих в решении задач, этапы
решения конкретной задачи полученным коллективом.}
\parskip=4pt
\begin{flushright}
\setlength{\parskip}{10pt}
    Received 13.03.2024
\end{flushright}
\end{multicols}
\newpage

\setlength{\parskip}{6pt}
\begin{center}
\chapter{ \leftskip=40mm { \huge\textbf{Towards the Theory of Semantic Space}}}
\vspace{0.3in}
    \newline
    
\large
 Valerian Ivashenko\\
  \textit{Department of Intelligent Information Technologies}\\
   \textit{Belarusion State University of Informatics and Radioelectronics
}\\
 Minsk, Republic of Belarus\\
    ivashenko@bsuir.by
\end{center}
\vspace{0.3in}
\begin{multicols}{2}
 \small \textbf{ \textit{ Abstract}—The paper considers models for investigating
the structure, topology and metric features of a semantic
space using unified knowledge representation.
\\ The classes of finite structures corresponding to ontological structures and sets of classical and non-classical kinds
are considered, and the enumerability properties of these
classes are investigated.
\\ The notion of operational-information space as a model
for investigating the interrelation of operational semantics
of ontological structures of large and small step is proposed.
\\ Quantitative features and invariants of ontological structures oriented to the solution of knowledge management
problems are considered.\\
\textit{Keywords}—Semantic Space, Neg-entropy, Operationa-linformation space, Enumerable sets, Natural numbers,
Ackermann coding, Generalized formal language, Enumerable self-founded Hereditarily finite sets, Countable nonidentically-equal Hereditarily finite sets, Multigraph, Hypergraph, Metagraph, Orgraph, Unoriented graph, Quasimetric, Orgraph invariant, Homomorphism, Isomorphism,
Homeomorhpism, Oriented sets, Graph wave-front, Dynamic graph system, Receptor, Effector, Transmitter, Resonator, Graph dimension, Fully-connected orgraph period,
Rado graph, Universal model, Stable structure, Operational
semantics, Denotational Semantics, Infinite structures, Generalized Kleene closure}
\begin{center}
  \big  I. Introduction
\end{center}
There are different approaches to the study of topological, metrical and other properties of signs in texts
leading to the consideration of corresponding semantic
(or meaning (sense)) spaces [56].
\newline Space is convenient because it is connected with some
ordinal or metric scale which allows to reduce the cost of
solving such cognitive tasks as searching (synthesis) or
checking (analysis) the presence of an element (including
for the purpose of eliminating redundancy) in a set
organized as a space.
\newline Knowledge integration based on unification is necessary both to eliminate redundancy and to compute semantic metrics. For this purpose, the developed model of
unified knowledge representation [1], [5] can be adopted.
\begin{center}
  \big II. Approaches to the construction of a meaning space
\end{center}
The history of the development of the concept of
“meaning space” and the corresponding models are described in the works [2], [11], [32], [56].\columnbreak
As stated in [56], the main approaches to the construction and research of the organization of meaning space
include:
\begin{itemize}[noitemsep]
    \item exterior studying the physical nature [30], [33], [48]
of processes including thinking processes [29],
    \item  (quantitative) interior using quantitative and soft
models, including probabilistic description of processes [11], [34], [35], [42], based on the practice
of using words of language [20], [53],
     \item (qualitative) interior investigating the structure of
represented knowledge and its dynamics [12], using
formal semiotic models [51].
\end{itemize}
In some cases, it is possible to combine these elements
of these approaches.
\par The following models and methods are used to construct and investigate the semantic space:
\begin{itemize}[noitemsep]
    \item  mathematical models of spaces [37]–[41], [43],
[44],
    \item   formal and generalized formal languages [45], [56],
     \item  methods of probability theory [11], [36], [54],
   \item methods of formal concepts analysis [58], [59],
   \item other models [3], [4], [45], [46], [49].

\end{itemize}
Further in the paper we consider the main classes of
structures, their attributes and corresponding types of
subspaces of the semantic space using unified knowledge
representation [5], [12].
\begin{center}
 \big III. Unified representation and classification of fully
representable finite knowledge structures 
\end{center}

At the level of syntax, using syntactic links, it is
possible to represent only finite knowledge structures in
a unified (explicit) way.
\par Let us consider the principles of unified representation
of knowledge [5], [12] with a structure that is one of finite
structures of different kinds. Let us compare a certain
class of structures to each kind of finite knowledge
structures.
\par Note that finite structures can be divided into two main
types: oriented finite structures and unoriented finite
structures [21].
\par The simplest unoriented finite structures are hereditarily finite sets [63]. The class of hereditarily finite sets
can be expressed as follows:
\par
\begin{center}
\begin{math}
\emptyset^{(+ \overset{*}{1})} = H_{\aleph_0}
\end{math}
\end{center}
\end{multicols}
\end{document}





